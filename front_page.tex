\documentclass[a4paper]{report}
\usepackage[utf8]{inputenc}
\usepackage[T1]{fontenc}
\usepackage{lmodern}
\usepackage{graphicx}
\usepackage{titlesec}
\usepackage{geometry} % Added for adjusting the page margins

\titleformat{\chapter}[display]
{\normalfont\huge\bfseries}{\chaptertitlename\ \thechapter}{20pt}{\Huge}

% Added for adjusting the page margins
\geometry{
    left=3cm,
    right=3cm,
    top=1cm,
    bottom=3cm
}

\pagenumbering{gobble}

\begin{document}

\begin{titlepage}
    \begin{center}
        \includegraphics[width=0.3\textwidth]{FRS-FNR_S2-PANT_UK_CS.png} % Insert your university logo here
        \vspace{1cm}
        
        \LARGE \textbf{University of Liège}\\
        \LARGE \textbf{Sorbonne University}\\
        
        \vspace{1.5cm}
        %{\huge \textbf{Doctoral Thesis}}\\[0.5cm]
        
        \hrulefill \\
        \vspace{0.5cm}
        {\huge \textbf{Towards a new insight of the carbon transport in the global ocean}} \\
        \hrulefill
        
        \vspace{1.5cm}
        {\Large \textbf{Florian Ricour}}

        \vspace{1cm}

        Dissertation presented in fulfillment for the degree of Doctor of Sciences
        
        \vfill

        \begin{flushleft}
            \large
            \textbf{Supervisors}\\
            \vspace{0.1cm}
            Hervé Claustre, director\\
            Marilaure Grégoire, director\\
            Alexander Barth, co-director\\
        \end{flushleft}
        
        \begin{flushleft}
            \large
            \textbf{Jury Members}\\
            \vspace{0.1cm}
            \begin{tabular}{lll}
                Pr. Stephanie Henson & University of Southampton & Reviewer\\
                Dr. Margaret Estapa & University of Maine & Reviewer \\
                Dr. Griet Neukermans & University of Ghent & Examiner \\
                Dr. Ivona Cetini\'{c} & Morgan State University & Examiner\\
                Pr. Lars Stemmann & Sorbonne University & Examiner\\
                Dr. Lionel Guidi & Sorbonne University & Guest\\
                Dr. Hervé Claustre & Sorbonne University & PhD Director\\
                Pr. Marilaure Grégoire & University of Liège & PhD Director\\
                Dr. Alexander Barth & University of Liège & PhD Co-director\\
            \end{tabular}
        \end{flushleft}
        
        \vspace{1cm}
        {\large May 2023} % Replace with the month and year of submission
   
    \end{center}
\end{titlepage}

\newpage
\thispagestyle{empty}
\null

\newpage
\thispagestyle{empty}
\null

\section*{Foreword}

As of 2022, it has now become obvious to the general public that the Earth's climate is changing. Massive flood events, repeated heatwaves, never-ending droughts, disappearing glaciers, biodiversity collapse, … Reports on weather anomalies were already published in the 1960's, the Intergovernmental Panel on Climate Change (IPCC) was created in 1988, the Kyoto Protocol was adopted in 1997, An Inconvenient Truth was released in 2006, the Paris Agreement was signed in 2015 and still, what has really changed?\\
\\
As I am writing this part, the COP27 (sponsored by Coca-Cola, no joke) is taking place with an acknowledged record number of fossil fuel lobbyists, bloody stadiums with outside air conditioning are about to be used in the desert for a month and it seems like 'leaders' of the Free World have been having hearing issues. See for yourself:\\
\\
\textit{"I am drowning"} said Pakistan\\
\\
\textit{"I'm sorry, what?"} said the Free World\\
\\
\textit{"I am starving"} said Ethiopia\\
\\
\textit{"Can't hear you, louder please!"} said the Free World\\
\\
\textit{"We are suffocating"} said India\\
\\
\textit{"Aaaah okay. Yes you're absolutely right, let's save the banks one more time"} said the Free World\\
\\
Although I am not an activist (yet?) or a politician (I am too honest to be elected), rest assured that there is no fake news in this work. I hope you will enjoy it.

\newpage
\thispagestyle{empty}
\null

\newpage
\thispagestyle{empty}
\null

\section*{Acknowledgements}

I would like to thank the members of the jury for taking the time to read and evaluate this three and a half year work. I look forward to discussing it during the defense.\\
\\
Merci aux membres du comité de thèse pour leur temps et leurs retours sur l'évolution du projet.\\
\\
Cette thèse a été particulière à bien des égards, avec son lot de points positifs et négatifs mais surtout avec son lot de rebondissements qui m'ont permis d'affûter mes connaissances, ma résilience et je l'espère ma tolérance. Pour cela, je tiens à remercier chaleureusement toutes les personnes y ayant contribué de près ou de loin.\\
\\
Merci à Hervé, Lionel et Jean-Olivier pour leur accueil au Laboratoire de Villefranche-sur-Mer et pour leur disponibilité ainsi que pour leurs conseils avisés, tant scientifiques que personnels.\\
\\
Merci à Alex, Charles et Arthur pour leur aide et leur gentillesse.\\
\\
Merci également aux personnes que j'ai pu côtoyer à COMPLEX comme chez OMTAB depuis mon passage en stage en 2018 et les multiples allers-retours Belgique-France.\\
\\
Merci à Antoine Mangin de m'avoir prêté un bureau dans les locaux d'ACRI sur Grasse l'été dernier pendant que mon appartement servait d'Airbnb estival pour le plus grand bonheur du portefeuille de mon proprio.\\
\\
Un grand merci à mes colocs de bureau: Ophélie, Laetitia et Thelma pour votre soutien et votre bonne humeur sans faille (contrairement à moi).\\
\\
Merci à AMP (elle se reconnaitra), Chloé et Anaïs pour les pauses midi sur le ponton. AMP, merci d'avoir accepté de vivre virtuellement avec moi afin que la sécu daigne enfin me donner une carte vitale !\\
\\
Ne me demandez pas des noms de restos ou de bars sur la côte d'Azur, je n'en ai aucune idée. Par contre, je connais pas mal les routes du coin. Merci à tous ceux qui ont partagé (ou subi) des kilomètres à mes côtés, surtout en montée (le plat pays, qui est le mien): Flavien, Ophé, Lucas, Thelma, Laeti et Satoshi.\\
\\
Un grand merci à Marine et Alice pour leur bienveillance. Marine, je te souhaite le meilleur pour tes projets futurs mais fais attention à qui tu donnes tes clés quand même.\\
\\
Merci également à Alex qui a été un excellent roommate pendant les confinements, couvre-feux et ordonnances en tout genre. See you in London !\\
\\
Merci aux Liégeois(es) (Thomas, Florent, Flop, Nico, Evgeny et Lucie) et en particulier Hadrien qui a accepté de m'avoir comme coloc pendant plusieurs mois (et qui y a survécu).\\
\\
Merci aux Montois (Mathieu, Chris, Flo, Ju \& Ju) pour votre soutien à distance, un mariage en 3 actes et pour vos visites sur Nice. Petite dédicace à Ju K qui, à force d'avoir joué avec le diable en subissant l'enfer des pourcentages, va littéralement en devenir un. Vivement 2024 pour casser la baraque !\\
\\
Un très grand merci à ma famille nucléaire qui ne comprend toujours pas mon job mais qui fait semblant quand je lui explique, c'est le principal. C'est vrai que la défense tombe mal au niveau timing donc on se voit à la Ducasse :-)\\
\\
Enfin, un grand merci à une personne toute particulière qui en a probablement marre que je parle de la Belgique mais qui à présent est parfaite bilingue, mission accomplie?

\newpage
\thispagestyle{empty}
\null

\newpage
\thispagestyle{empty}
\null

\section*{Summary}

\textbf{Towards a new insight of the carbon transport in the global ocean}\\
\\
The ocean is known to play a key role in the carbon cycle. Without it, atmospheric CO$_{2}$ levels would be much higher than they are today thanks to the presence of carbon pumps that maintain a gradient of dissolved inorganic carbon (DIC) between the surface and the deep ocean. The biological carbon pump (BCP) is primarily responsible for this gradient. It consists in a series of ocean processes through which inorganic carbon is fixed as organic matter by photosynthesis in sunlit surface waters and then transported to the ocean interior and possibly the sediment where it will be sequestered from the atmosphere for millions of years. The BCP was long thought as solely the gravitational settling of particulate organic carbon (POC). However, a new paradigm for the BCP has recently been defined in which physically and biologically mediated particle injection pumps have been added to the original definition. Physically mediated particle injection pumps provide a pathway to better understand the transport of dissolved organic carbon (DOC) whereas biologically mediated particle injection pumps focus on the transport of POC by vertically migrating animals, either daily or seasonally. Therefore, a better understanding of these processes could help bridge the gap between carbon leaving the surface and carbon demand in the ocean interior. To address this new paradigm, this work will benefit from the advent of recent sensors that equip a new generation of Biogeochemical-Argo floats (BGC-Argo). The first part focuses on the development of an embedded zooplankton classification model for the Underwater Vision Profiler 6 (UVP6) under strict technical and energy constraints. The second part studies particle and carbon fluxes in the Labrador Sea using BGC-Argo floats equipped for the first time with the UVP6 and an optical sediment trap (OST), providing two independent measurements of sinking particles. The last part consists in revisiting the BCP using a new framework called CONVERSE for Continuous Vertical Sequestration. With this new approach, we re-evaluate the total carbon sequestered from the atmosphere ($\geq$ 100 years) by the BCP and its transport pathways on the entire water column, in contrast to the carbon sequestration typically assumed below a fixed reference depth.\\
\\
\textbf{Keywords} Carbon pumps -- Underwater Vision Profiler 6 -- BGC-Argo floats -- Machine learning

\newpage

\section*{Résumé}

\textbf{Vers une meilleure compréhension du transport du carbone dans l'océan global}\\
\\
L'océan est connu pour jouer un rôle clé dans le cycle du carbone. Sans lui, les niveaux de CO$_{2}$ atmosphérique seraient bien plus élevés qu'aujourd'hui grâce à la présence de pompes à carbone qui maintiennent un gradient de carbone inorganique dissous (DIC) entre la surface et le fond de l'océan. La pompe à carbone biologique (BCP) est principalement responsable de ce gradient. Elle consiste en une série de processus océaniques au cours desquels le carbone inorganique est converti en matière organique via la photosynthèse dans les eaux de surface, puis transporté vers l'intérieur de l'océan et éventuellement les sédiments où il sera séquestré par rapport à l'atmosphère pour des millions d'années. La BCP était longtemps considérée comme étant uniquement la déposition gravitationnelle de carbone organique particulaire (POC). Cependant, un nouveau paradigme pour la BCP a récemment été défini dans lequel des pompes d'origine physique et biologique d'injection de particules ont été ajoutées à la définition originale. Les pompes physiques d'injection de particules fournissent un moyen de mieux comprendre le transport de carbone organique dissous (DOC), tandis que les pompes biologique d'injection de particules se concentrent sur le transport de POC par des animaux migrant verticalement, quotidiennement ou saisonnièrement. Par conséquent, une meilleure compréhension de ces processus pourrait aider à combler l'écart entre le carbone quittant la surface et la demande de carbone dans l'océan profond. Pour aborder ce nouveau paradigme, ce travail bénéficiera de l'arrivée de capteurs récents équipant une nouvelle génération de flotteurs Biogéochimiques-Argo (BGC-Argo). La première partie se concentre sur le développement d'un modèle embarqué de classification de zooplancton pour l'Underwater Vision Profiler 6 (UVP6) avec des contraintes techniques et énergétiques strictes. La deuxième partie étudie les flux de particules et de carbone dans la mer du Labrador en utilisant des flotteurs BGC-Argo équipés pour la première fois de l'UVP6 et d'un piège à sédiments optique (OST), fournissant deux mesures indépendantes des particules. La dernière partie consiste à revisiter la BCP en utilisant un nouveau cadre appelé CONVERSE qui fait référence à la séquestration verticale continue du carbone dans la colonne d'eau. Avec cette nouvelle approche, nous réévaluons le carbone total séquestré par rapport à l'atmosphère ($\geq$ 100 ans) par la BCP et ses voies de transport sur toute la colonne d'eau, par opposition à la séquestration de carbone généralement supposée en-dessous d'une profondeur de référence fixe.\\
\\
\textbf{Mots clés} Pompes à carbone -- Underwater Vision Profiler 6 -- Flotteurs BGC-Argo -- Machine learning

\newpage
\thispagestyle{empty}
\null


\end{document}

